




\documentclass[12pt, letterpaper]{article}
\usepackage{graphicx}
\usepackage{hyperref}
\usepackage[ngerman]{babel}


\title{Notizen zum Projekt}
\author{Kevin P. Bernowski}
\date{\today}




\begin{document}

\maketitle
\tableofcontents
\section{Risiko-Simulator}
    \subsection{Risiko}

    Die Spielregeln und der Spielverlauf werden in der README-Datei des Git-Projektes erklärt: \url{https://github.com/berkev/risiko-simulator}.

    
    \subsection{Modellierung}

    Ein Spielbrett ist ein bidirektionaler Graph 
    \[G := (V,E), \hspace{1cm} E\subset V\times V.\]
    Die Knoten in 
    \[V :=\{0,1,...,N\}, \hspace{1cm} N \in \mathbf{N} \]
    werden im folgenden auch \textbf{Gebiete} genannt.
    \newline
    Die Truppenstärke und die Farbe zu einer gegebenen Farbmenge $\mathcal{F}$ zum Zeitpunkt $n\in\mathbf{N}$ werden definiert als
    \[ t_{n}:V\rightarrow \mathbf{N}_{>0},
    \] und 
    \[ f_{n}:V\rightarrow \mathcal{F}.\]
    \newline
    Ein Spiel(-verlauf) der Länge $L\in\mathbf{N}$ mit maximaler Truppenreserve $K\in\mathbf{N}$ auf einem Spielbrett $G$ gegeben die Farbmenge $\mathcal{F}$  bezeichnet eine Folge von Tupeln 
    \[((t_{n},f_{n}))_{n=0}^{L}\]
    mit den folgenden Bedingungen:
    \begin{enumerate}
        \item Ein Spieler kann höchstens $K$ Truppen auf seine Gebiete verteilt haben:
        \[\forall c\in\mathcal{F},n\leq L: \sum_{k\in V, f_{n}(k)=c} \leq K\]
        \item keine unbesetzten Gebiete:
        \[\forall v \in V,n\leq L: t_{n}(v)>0\]
        \item Einmal raus, immer raus:
        \[\forall c\in \mathcal{F}:\]
        \[\exists i\leq L: c \notin \{f_{i}(v) : v\in V\} \Rightarrow c \notin \{f_{j}(v) : v\in V\}\hspace{5mm} \forall j=i,...,L.\]
    \end{enumerate}
    Die dazugehörige Menge aller Spielverläufe schreiben wir als
    \[ S(G,K,\mathcal{F}) \]
    und die Teilmenge aller Spiele der Länge $l\in\mathbf{N}$ wird bezeichnet durch
    \[S_{l}(G,K,\mathcal{F}).\]
        

    

    

\section{Analyse von Gewinnstrategien}


\end{document}